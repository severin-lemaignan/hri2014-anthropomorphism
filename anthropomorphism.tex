\documentclass{acm_proc_article-sp}

% UTF8 support
\usepackage[utf8x]{inputenc}
\usepackage[T1]{fontenc}


\begin{document}

\title{On the Dynamics of Anthropomorphism in Robotics
}

\numberofauthors{3} 
\author{
\alignauthor
Séverin Lemaignan\\
    \affaddr{Computer-Human Interaction in Learning and Instruction (CHILI)}\\
    \affaddr{Ecole Polytechnique Fédérale de Lausanne (EPFL)}\\
    \affaddr{CH-1015 Lausanne, Switzerland}\\
    \email{severin.lemaignan@epfl.ch}
\alignauthor
Julia Fink\\
    \affaddr{Computer-Human Interaction in Learning and Instruction (CHILI)}\\
    \affaddr{Ecole Polytechnique Fédérale de Lausanne (EPFL)}\\
    \affaddr{CH-1015 Lausanne, Switzerland}\\
    \email{julia.fink@epfl.ch}
\alignauthor
Pierre Dillenbourg\\
    \affaddr{Computer-Human Interaction in Learning and Instruction (CHILI)}\\
    \affaddr{Ecole Polytechnique Fédérale de Lausanne (EPFL)}\\
    \affaddr{CH-1015 Lausanne, Switzerland}\\
    \email{pierre.dillenbourg@epfl.ch}
}
\date{30 July 1999}

\maketitle

\begin{abstract}

While anthropomorphism in robotics is a commonly discussed trait of
HRI, it paradoxically lacks formal grounds. This article
attempts first at providing a comprehensive synthesis of the social
phenomenon of \textit{perceiving human-like characteristics in non-human
agents and attributing those characteristics to robots}. We draw on
social sciences and psychology, as well as on a critical survey of
existing literature in the HRI community, to ground the concept of
\textit{anthropomorphism}. We then suggest to go beyond the traditional
perception of anthropomorphism as a static feature of a system: we propose to
understand anthropomorphism as a dynamic, non-monotonic and
context-dependent process, which evolves over time and accounts for
special events such as the so-called \textit{novelty effect}. To this end,
we introduce a model of anthropomorphism that analyzes the phenomenon
along three interaction phases. These findings are supported by a long-term
study conducted in a real human environment.
\end{abstract}

\terms{Experimentation, Human Factors}

\keywords{Anthropomorphism, Design, Human-Robot Interaction, Social Issues in Robotics, Acceptance of Robots}

\bibliographystyle{abbrv}
\bibliography{biblio} 
%Generated by bibtex from your ~.bib file.  Run latex,
%then bibtex, then latex twice (to resolve references)
%to create the ~.bbl file.  Insert that ~.bbl file into
%the .tex source file and comment out
%the command \texttt{{\char'134}thebibliography}.
\balancecolumns
\end{document}
